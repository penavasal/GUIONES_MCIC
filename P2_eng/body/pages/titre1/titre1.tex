\section{Introduction to the steady-state diffusion problem}

\subsection{The problem of steady-state flow in a porous medium}

Next, as a review, we summarize the most important aspects of the
steady-state flow problem in a porous medium of an incompressible fluid:
\begin{itemize}
\item The fluid flow in a porous medium takes place when there is an
  energy difference between two points of the fluid. The flow takes
  place from the point of higher energy (upstream) to the point of
  less energy (downstream).
\item The energy level of an incompressible fluid at a point for a
  steady state is expressed by the Bernoulli's Equation, which
  calculates the total head of a fluid $h$ at a point as the sum of
  three components: pressure head, head of elevation and velocity
  head:
  \begin{equation}
    \label{eq:0101}
    h=h_p+h_e+h_v=p/\gamma_w+z+v^2/(2g)\approx p/\gamma_w+z
  \end{equation}
  being $p$ the fluid pore pressure, $\gamma_w$ its specific weight,
  $z$ its elevation head measured with respect to a selected
  horizontal line (the points under this line will have a negative
  elevation head). The velicity head has been neglected considering
  the velocity of the fluid is negligible compared to the other terms.
\item When the fluid passes from a point A with total head $ h_A $ to
  a point B with total head $ h_B <h_A $, there is a loss of energy
  due to the friction to the flow offered by the soil. The hydraulic
  gradient $ i_h $ is defined as the total head loss per unit
  length. If the distance between A and B according to their current
  line is $ L_{AB}$ the mean hydraulic gradient between these two
  points is:
  \begin{equation}
    \label{eq:0102}
    i_h=\frac{h_B-h_A}{L_{AB}}=\frac{(p_B/\gamma_w+z_B)-(p_A/\gamma_w+z_A)}{L_{AB}}
  \end{equation}
  And the hydraulic gradient at a point is:
  \begin{equation}
    \label{eq:0103}
    \textbf{i}_h=\bm{\nabla}h
  \end{equation}
\item When the flow through the soil is laminar, Darcy's law is
  applied (equivalent to the Fick's law for the diffusion problem
  explained on page 5 on the presentation of the theory class by
  making $u=h$ and $\bm{C}=\bm{K}$):
  \begin{equation}
    \label{eq:0104}
    \bm{q}=-\bm{K}\cdot \bm{\nabla}h
  \end{equation}
  Where $\bm{q}$ is the flow vector (its units are fluid volume per
  unit area and per unit time).
\item The balance equation of the laminar (stationary) flow problem in
  a porous medium of an incompressible fluid is equivalent to the
  (stationary) Diffusion Equation described on page 9 of the
  presentation of the theory class.
\end{itemize}

\subsection{Abaqus modeling strategy of the flow in a porous medium
  using the thermal problem}

In Abaqus the problem of fluid flow in a porous medium is within a
coupled formulation that solves:
\begin{itemize}
\item The mechanical problem of soil deformation (which can be
  considered rigid or deformable)
\item The problem of fluid transport within the soil
\end{itemize}

Abaqus has also implemented the heat conduction problem, which
formally has the same formulation as the problem of fluid flow in a
porous medium (both problems are idealized with the equation in
partial derivatives that we have called Diffusion Equation in theory
class, see page 12). We can strategically use the Abaqus modulus of heat conduction
to reproduce the fluid flow in a porous medium by making the following
equivalence:
\begin{table}[!h]
  \centering
  \begin{tabular}{cll}
    \hline
    Parameters&Heat Conduction & Flow in porous medium\\
    \hline
    $h$&Temperature  &Total (hydraulic) head    \\
    $k$&Conductivity coefficient& Permeability coefficient  \\
    \hline
  \end{tabular}
  \caption{Equivalence thermal problem - flow problem in porous medium}
  \label{tab:101}
\end{table}

Therefore, in this practice we will solve the problem of fluid
transport in a porous medium using the Abaqus heat conduction module.



\hspace{20mm}\hrulefill$\star$\hrulefill\hspace{20mm}
