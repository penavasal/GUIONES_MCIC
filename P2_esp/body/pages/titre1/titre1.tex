\section{Introducción al problema de difusión estacionario}

\subsection{El problema de flujo estacionario en un medio poroso}

A continuación, y a modo de repaso, se resumen los puntos más
importantes del problema de flujo estacionario en un medio poroso de
un fluido incompresible:
\begin{itemize}
\item El flujo de un fluido en medios porosos ocurre cuando hay una
  diferencia de energía entre dos puntos del fluido. El flujo tiene
  lugar desde el punto de mayor energía (aguasarriba) al punto de
  menor energía (aguasabajo).
\item El nivel de energía de un fluido incompresible en un punto para
  un régimen estacionario viene representado por la Ecuación de
  Bernoulli, que calcula la altura total o piezométrica del fluido $h$
  en un punto como la suma de tres componentes: altura de presión,
  altura de elevación y altura de velocidad:
  \begin{equation}
    \label{eq:0101}
    h=h_p+h_e+h_v=p/\gamma_w+z+v^2/(2g)\approx p/\gamma_w+z
  \end{equation}
  siendo $p$ la presión de poro del fluido, $\gamma_w$ su peso
  específico, $z$ su altura de elevación medida respecto a una línea
  horizontal seleccionada (los puntos bajo dicha línea tendrán una
  altura de elevación negativa), y habiéndose despreciado la altura de
  velocidad por considerar la velocidad del fluido $v$ despreciable
  frente a los otros términos.
\item Cuando el fluido pasa de un punto A con altura total $h_A$ a un
  punto B con altura total $h_B<h_A$ se produce una pérdida de energía
  debido a la fricción al flujo que ofrece el suelo. Se define el
  gradiente hidráulico $i_h$ como la pérdida de altura total por
  unidad de longitud. Si la distancia entre A y B según su línea de
  corriente es $L_{AB}$ el gradiente hidráulico medio entre esos dos
  puntos es:
  \begin{equation}
    \label{eq:0102}
    i_h=\frac{h_B-h_A}{L_{AB}}=\frac{(p_B/\gamma_w+z_B)-(p_A/\gamma_w+z_A)}{L_{AB}}
  \end{equation}
  y el gradiente hidráulico en un punto es:
  \begin{equation}
    \label{eq:0103}
    \textbf{i}_h=\bm{\nabla}h
  \end{equation}
\item Cuando el flujo a través del suelo es laminar se aplica la ley
  de Darcy (equivalente a la ley de Fick para el problema de difusión
  explicado en la página 5 de la presentación de la clase de teoría
  haciendo $u=h$ y $\bm{C}=\bm{K}$):
  \begin{equation}
    \label{eq:0104}
    \bm{q}=-\bm{K}\cdot \bm{\nabla}h
    \end{equation}
    siendo $\bm{q}$ el vector flujo (sus unidades son volumen de
    fluido por unidad de área y por unidad de tiempo).
  \item La ecuación de balance del problema del flujo laminar
    (estacionario) en un medio poroso de un fluido incompresible es
    equivalente a la Ecuación de Difusión (estacionaria) descrita en
    la página 8 de la presentación de la clase de teoría.
  \end{itemize}

\subsection{Estrategia de modelización en Abaqus del flujo en un medio
  poroso usando el problema térmico}

En Abaqus el problema de flujo de un fluido en un medio poroso está
dentro de una formulación acoplada que resuelve:
\begin{itemize}
\item El problema mecánico de deformación del suelo (que podemos
  considerarlo rígido o deformable)
\item El problema de transporte de un fluido dentro del suelo
\end{itemize}

Abaqus también tiene implementado el problema de conducción de calor,
que formalmente tiene la misma formulación que el problema de flujo de
un fluido en un medio poroso (ambos problemas se idealizan con la
ecuación en derivadas parciales que hemos llamado Ecuación de Difusión
en clase de teoría). Estratégicamente podemos usar el módulo de Abaqus
de conducción de calor para reproducir el flujo de un fluido en un
medio poroso haciendo la siguiente equivalencia:

\begin{table}[!h]
  \centering
  \begin{tabular}{cll}
    \hline
    Parámetros&Conducción de Calor & Flujo medio poroso\\
    \hline
    $h$&Temperatura  &Altura hidráulica    \\
    $k$&Coeficiente de conductividad & Coeficiente permeabilidad  \\
    \hline
  \end{tabular}
  \caption{Equivalencia problema térmico - problema flujo en medio poroso}
  \label{tab:101}
\end{table}

Por lo tanto, en esta práctica resolveremos el problema de transporte
de un fluido dentro de un medio poroso usando el módulo de conducción
de calor de Abaqus.




\hspace{20mm}\hrulefill$\star$\hrulefill\hspace{20mm}
